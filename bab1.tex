\fancyhf{} 
\fancyfoot[C]{\thepage}

\chapter{PENDAHULUAN}

\section{\uppercase{LATAR BELAKANG}}
Usaha Mikro, Kecil, dan Menengah (UMKM) merupakan salah satu penggerak roda perekonomian suatu rakyat yang tangguh. Usaha Mikro, Kecil, dan Menengah (UMKM) mempunyai peranan yang penting dalam pertumbuhan ekonomi dan industri suatu Negara \citep{prastika2014pengaruh}. Hal ini karena usaha tersebut merupakan tulang punggung sistem ekonomi kerakyatan yang tidak hanya ditujukan untuk mengurangi masalah kesenjangan antar golongan pendapatan dan antar pelaku usaha, ataupun pengentasan kemiskinan dan penyerapan tenaga kerja.

\par Perkembangan teknologi yang sangat pesat di era globalisasi saat ini telah memberikan banyak manfaat dalam berbagai aspek kehidupan, termasuk dibidang UMKM. Di mana para pelaku UMKM sekarang ini dapat memanfaatkan kemajuan teknologi tersebut untuk memasarkan produknya secara digital. Pemasaran digital adalah pemasaran yang memanfaatkan akses internet, media sosial, maupun perangkat digital lainnya \citep{hardilawati2020strategi}. Dengan melakukan pemasaran secara digital dapat menjangkau pelanggan yang lebih luas lagi dan mempromosikan produk-produk kepada calon pembeli baru. Melihat banyaknya keuntungan dari pemasaran secara digital membuat para pelaku usaha tanaman hidroponik yang ada di Banda Aceh tertarik untuk memasarkan produknya secara digital.

\par Hidroponik merupakan cara bercocok tanam yang tidak menggunakan tanah sebagai media tanam, tetapi hanya menggunakan air yang mengandung nutrisi yang diperlukan tanaman \citep{prayitno2017sistem}. Pelaku UMKM hidroponik ini, sebenarnya sudah memasarkan produknya secara digital lewat aplikasi sosial media seperti WhatsApp dan Instagram, hanya saja penjualannya dinilai masih kurang efektif karena pelanggannya hanya berasal dari orang yang mengetahui kontak dan sosial media mereka saja, belum lagi pelanggan juga tidak dapat mengetahui ketersediaan produknya. Berangkat dari permasalahan tersebut pihak UMKM hidroponik berencana untuk membuat sebuah aplikasi khusus yang bertindak sebagai \textit{e-commerce} agar mempermudah proses transaksi antara penjual dan pembeli, juga untuk menjaga harga produk antar penjual serta meningkatkan kepercayaan dari pelanggan.

\par Berdasarkan uraian diatas, maka dibentuklah sebuah program pengabdian masyarakat yang terdiri dari dosen, mahasiswa dan mitra pengusaha tanaman hidroponik. Program pengabdian masyarakat ini merupakan kolaborasi antara program studi Informatika USK dengan program studi Agribisnis USK dan bekerjasama dengan perusahaan mitra yaitu Ruhul Hidroponik dan Ismulia Farm. \textit{Output} dari program ini berupa aplikasi jual beli tanaman hidroponik yang dapat digunakan oleh para penjual dan pembeli. Hal tersebut yang melatarbelakangi dibangunnya aplikasi penjualan tanaman hidroponik berbasis web. Aplikasi ini nantinya dapat digunakan oleh admin dan penjual. Sedangkan untuk pembeli akan dibangun aplikasi berbasis android oleh rekan setim penulis yaitu Yaumil Aghnia. Pembuatan aplikasi penjualan berbasis web ini dibangun menggunakan \textit{framework} Laravel dan MySQL sebagai \textit{database}. Selain itu, juga akan dibuatkan REST API dari aplikasi web tersebut untuk dijadikan sebagai \textit{backend} pada aplikasi Android. Dengan adanya aplikasi ini diharapkan dapat membantu kelompok penjual tanaman hidroponik memasarkan produknya dengan lebih mudah dan efesien.

\fancyhf{} 
\fancyfoot[R]{\thepage}

\section{\uppercase{RUMUSAN MASALAH}}
Berdasarkan latar belakang di atas, permasalahan dalam penelitian ini dapat dirumuskan sebagai berikut:
\begin{enumerate}
	\item Bagaimana merancang dan membangun aplikasi penjualan tanaman hidroponik berbasis web yang dapat digunakan oleh admin dan penjual?
	\item Bagaimana mengimplementasikan Laravel sebagai \textit{framework} yang digunakan untuk membangun aplikasi penjualan tanaman hidroponik berbasis web?
	\item Bagaimana membangun REST API dari aplikasi berbasis web untuk digunakan sebagai \textit{backend}  pada aplikasi berbasis Android?
	\item Bagaimana menguji fungsionalitas aplikasi menggunakan metode \textit{Black Box Testing} dan menganalisis \textit{usability} aplikasi menggunakan metode \textit{Usability Metric for User Experience} (UMUX)?
\end{enumerate}

\section{\uppercase{TUJUAN PENELITIAN}}
Berdasarkan rumusan masalah yang telah disebutkan sebelumnya, maka dapat dipaparkan tujuan dari penelitian ini adalah sebagai berikut:
\begin{enumerate}
	\item Merancang dan membangun aplikasi penjualan tanaman hidroponik berbasis web yang dapat digunakan oleh admin dan penjual.
	\item Mengimplementasikan Laravel sebagai \textit{framework} yang digunakan untuk membangun aplikasi penjualan tanaman hidroponik berbasis web.
	\item Membangun REST API dari aplikasi berbasis web untuk digunakan sebagai \textit{backend} pada aplikasi berbasis Android.
	\item  Menguji fungsionalitas aplikasi menggunakan metode \textit{Black Box Testing} dan menganalisis \textit{usability} aplikasi menggunakan metode \textit{Usability Metric for User Experience} (UMUX).
\end{enumerate}


\section{\uppercase{MANFAAT PENELITIAN}}
Adapun manfaat dari penelitian ini adalah sebagai berikut:
\begin{enumerate}
	\item Mempermudah admin dalam mengelola data yang ada di aplikasi.
	\item Mempermudah para penjual tanaman hidroponik Banda Aceh untuk memasarkan produknya secara digital lewat aplikasi.
	\item Memberikan kemudahan untuk pembuatan aplikasi berbasis web dengan menggunakan \textit{framework} Laravel
	\item Memungkinkan pembuatan aplikasi penjualan tanaman hidroponik berbasis web dan aplikasi berbasis android saling terhubung karena telah menggunakan REST API.
\end{enumerate}

% Baris ini digunakan untuk membantu dalam melakukan sitasi
% Karena diapit dengan comment, maka baris ini akan diabaikan
% oleh compiler LaTeX.
\begin{comment}
\bibliography{daftar-pustaka}
\end{comment}