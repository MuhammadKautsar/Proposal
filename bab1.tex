\fancyhf{} 
\fancyfoot[C]{\thepage}

\chapter{PENDAHULUAN}

\section{\uppercase{LATAR BELAKANG}}
UMKM (Usaha Mikro Kecil Menengah) adalah unit usaha produktif yang berdiri sendiri, yang dilakukan oleh orang perorangan atau Badan usaha di semua sektor ekonomi \citep{tambunan2012peluang}. UMKM sering disebut sebagai salah satu pilar kekuatan perekonomian suatu daerah. Hal ini disebabkan karena UMKM mempunyai fleksibilitas dan kemampuan menyesuaikan diri terhadap kondisi pasar yang berubah dengan cepat dibanding dengan perusahaan skala besar \citep{sartika2002ekonomi}. Hal itulah yang membuat UMKM dapat bertahan walaupun dalam kondisi pandemi seperti sekarang ini.

\par Perkembangan teknologi yang sangat pesat di era globalisasi saat ini telah memberikan banyak manfaat dalam berbagai aspek kehidupan, termasuk dibidang UMKM. Dimana para pelaku UMKM sekarang ini dapat memanfaatkan kemajuan teknologi tersebut untuk memasarkan produknya secara digital. Pemasaran digital adalah pemasaran yang memanfaatkan akses internet, media sosial, maupun perangkat digital lainnya \citep{hardilawati2020strategi}. Dengan melakukan pemasaran secara digital dapat menjangkau pelanggan yang lebih luas lagi dan mempromosikan produk-produk kepada calon pembeli baru. Melihat banyaknya keuntungan dari pemasaran secara digital membuat para pelaku usaha tanaman hidroponik yang ada di Banda Aceh pun tertarik untuk memasarkan produknya secara digital.

\par Hidroponik adalah sistem penanaman tanaman tanpa menggunakan media tanam tanah dan menggunakan larutan nutrisi yang mengandung garam organik untuk menumbuhkan perakaran yang ideal \citep{rosliani2005budidaya}. Pelaku UMKM hidroponik ini, sebenarnya sudah memasarkan produknya secara digital lewat aplikasi sosial media seperti WhatsApp dan Instagram, hanya saja penjualannya dinilai masih kurang efektif karena pelanggannya hanya berasal dari orang yang mengetahui kontak dan sosial media mereka saja, belum lagi mengenai ketersedian produknya yang harus ditanyakan terlebih dahulu kepada penjualnya. Berangkat dari permasalahan tersebut pihak UMKM hidroponik berencana untuk memasarkan produknya lewat aplikasi khusus yang bertindak sebagai \textit{e-commerce} agar mempermudah proses transaksi antara penjual dan pembeli, juga diharapkan dapat meningkatkan trafik penjualannya.

\par Berdasarkan uraian diatas, penulis disini bertugas untuk merancang dan membangun sebuah aplikasi berbasis web untuk admin dan seller. Aplikasi ini nantinya akan diintegrasikan dengan aplikasi berbasis android untuk melakukan pembelian produk tanaman hidroponik. Pembuatan aplikasi penjualan berbasis web ini dibangun menggunakan framewok Laravel dan MySQL sebagai databasenya. Selain itu, juga akan dibuatkan REST API dari aplikasi web tersebut untuk dijadikan sebagai backend pada aplikasi android. 

\fancyhf{} 
\fancyfoot[R]{\thepage}

\section{\uppercase{RUMUSAN MASALAH}}
Berdasarkan latar belakang di atas, permasalahan dalam penelitian ini dapat dirumuskan sebagai berikut:
\begin{enumerate}
	\item Bagaimana merancang dan membangun aplikasi penjualan tanaman hidroponik berbasis web untuk admin dan seller.
	\item Bagaimana mengimplementasikan Laravel sebagai \textit{framework} yang digunakan untuk membangun aplikasi penjualan tanaman hidroponik berbasis Web.
	\item Bagaimana membangun REST API dari aplikasi berbasis web untuk digunakan sebagai backend  pada aplikasi berbasis Android.
\end{enumerate}

\section{\uppercase{TUJUAN PENELITIAN}}
Berdasarkan rumusan masalah yang telah disebutkan sebelumnya, maka dapat dipaparkan tujuan dari penelitian ini adalah sebagai berikut:
\begin{enumerate}
	\item Merancang dan membangun aplikasi penjualan tanaman hidroponik berbasis web untuk admin dan seller.
	\item Mengimplementasikan Laravel sebagai \textit{framework} yang digunakan untuk membangun aplikasi penjualan tanaman hidroponik berbasis Web.
	\item Membangun REST API dari aplikasi berbasis web untuk digunakan sebagai backend pada aplikasi berbasis Android.
\end{enumerate}


\section{\uppercase{MANFAAT PENELITIAN}}
Adapun manfaat dari penelitian ini adalah sebagai berikut:
\begin{enumerate}
	\item Mempermudah admin dalam mengelola aplikasi.
	\item Memudahkan pelaku usaha hidroponik dalam menjual dan mengelola produk yang dijual lewat aplikasi.
	\item Terintegrasi dengan aplikasi mobile agrihub, sehingga memudahkan pelanggan untuk membeli produk tanaman hidroponik.
	
\end{enumerate}

% Baris ini digunakan untuk membantu dalam melakukan sitasi
% Karena diapit dengan comment, maka baris ini akan diabaikan
% oleh compiler LaTeX.
\begin{comment}
\bibliography{daftar-pustaka}
\end{comment}