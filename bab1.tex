\fancyhf{} 
\fancyfoot[C]{\thepage}

\chapter{PENDAHULUAN}

\section{\uppercase{LATAR BELAKANG}}
Usaha Mikro, Kecil, dan Menengah (UMKM) merupakan salah satu penggerak roda perekonomian suatu rakyat yang tangguh. Usaha Mikro, Kecil, dan Menengah (UMKM) mempunyai peranan yang penting dalam pertumbuhan ekonomi dan industri suatu Negara \citep{prastika2014pengaruh}. Hal ini selain karena usaha tersebut merupakan tulang punggung sistem ekonomi kerakyatan yang tidak hanya ditujukan untuk mengurangi masalah kesenjangan antar golongan pendapatan dan antar pelaku usaha, ataupun pengentasan kemiskinan dan penyerapan tenaga kerja. Lebih dari itu, pengembangannya mampu memperluas bisnis ekonomi dan dapat memberikan kontribusi dalam mempercepat perubahan struktural, yaitu meningkatnya perekonomian daerah dan ketahanan ekonomi nasional.

\par Perkembangan teknologi yang sangat pesat di era globalisasi saat ini telah memberikan banyak manfaat dalam berbagai aspek kehidupan, termasuk dibidang UMKM. Di mana para pelaku UMKM sekarang ini dapat memanfaatkan kemajuan teknologi tersebut untuk memasarkan produknya secara digital. Pemasaran digital adalah pemasaran yang memanfaatkan akses internet, media sosial, maupun perangkat digital lainnya \citep{hardilawati2020strategi}. Dengan melakukan pemasaran secara digital dapat menjangkau pelanggan yang lebih luas lagi dan mempromosikan produk-produk kepada calon pembeli baru. Melihat banyaknya keuntungan dari pemasaran secara digital membuat para pelaku usaha tanaman hidroponik yang ada di Banda Aceh pun tertarik untuk memasarkan produknya secara digital.

\par Hidroponik merupakan cara bercocok tanam yang tidak menggunakan tanah sebagai media tanam, tetapi hanya menggunakan air yang mengandung nutrisi yang diperlukan tanaman \citep{prayitno2017sistem}. Pelaku UMKM hidroponik ini, sebenarnya sudah memasarkan produknya secara digital lewat aplikasi sosial media seperti WhatsApp dan Instagram, hanya saja penjualannya dinilai masih kurang efektif karena pelanggannya hanya berasal dari orang yang mengetahui kontak dan sosial media mereka saja, belum lagi mengenai ketersediaan produknya yang harus ditanyakan terlebih dahulu kepada penjualnya. Berangkat dari permasalahan tersebut pihak UMKM hidroponik berencana untuk memasarkan produknya lewat aplikasi khusus yang bertindak sebagai \textit{e-commerce} agar mempermudah proses transaksi antara penjual dan pembeli, juga diharapkan dapat meningkatkan angka penjualannya.

\par Berdasarkan uraian di atas, penulis akan merancang dan membangun sebuah aplikasi berbasis web untuk admin dan penjual. Aplikasi ini nantinya akan diintegrasikan dengan aplikasi berbasis Android untuk melakukan pembelian produk tanaman hidroponik. Pembuatan aplikasi penjualan berbasis web ini dibangun menggunakan \textit{framework} Laravel dan MySQL sebagai \textit{database}. Selain itu, juga akan dibuatkan REST API dari aplikasi web tersebut untuk dijadikan sebagai \textit{backend} pada aplikasi Android.

\fancyhf{} 
\fancyfoot[R]{\thepage}

\section{\uppercase{RUMUSAN MASALAH}}
Berdasarkan latar belakang di atas, permasalahan dalam penelitian ini dapat dirumuskan sebagai berikut:
\begin{enumerate}
	\item Bagaimana merancang aplikasi penjualan tanaman hidroponik berbasis web untuk admin dan penjual?
	\item Bagaimana mengimplementasikan Laravel sebagai \textit{framework} yang digunakan untuk membangun aplikasi penjualan tanaman hidroponik berbasis web?
	\item Bagaimana membangun REST API dari aplikasi berbasis web untuk digunakan sebagai \textit{backend}  pada aplikasi berbasis Android?
	\item Bagaimana menganalis kelayakan aplikasi yang sudah dibangun?
\end{enumerate}

\section{\uppercase{TUJUAN PENELITIAN}}
Berdasarkan rumusan masalah yang telah disebutkan sebelumnya, maka dapat dipaparkan tujuan dari penelitian ini adalah sebagai berikut:
\begin{enumerate}
	\item Merancang aplikasi penjualan tanaman hidroponik berbasis web untuk admin dan penjual.
	\item Mengimplementasikan Laravel sebagai \textit{framework} yang digunakan untuk membangun aplikasi penjualan tanaman hidroponik berbasis web.
	\item Membangun REST API dari aplikasi berbasis web untuk digunakan sebagai \textit{backend} pada aplikasi berbasis Android.
	\item  Menganalisis fungsionalitas aplikasi menggunakan metode \textit{Black Box Testing} dan menganalisis \textit{usability} aplikasi menggunakan metode \textit{Usability Metric for User Experience} (\textit{UMUX}).
\end{enumerate}


\section{\uppercase{MANFAAT PENELITIAN}}
Adapun manfaat dari penelitian ini adalah sebagai berikut:
\begin{enumerate}
	\item Mempermudah admin dalam mengelola aplikasi.
	\item Memudahkan pelaku usaha hidroponik dalam menjual dan mengelola produk yang dijual lewat aplikasi.
	\item Terintegrasi dengan aplikasi \textit{mobile} AgriHub, sehingga memudahkan pelanggan untuk membeli produk tanaman hidroponik.
	
\end{enumerate}

% Baris ini digunakan untuk membantu dalam melakukan sitasi
% Karena diapit dengan comment, maka baris ini akan diabaikan
% oleh compiler LaTeX.
\begin{comment}
\bibliography{daftar-pustaka}
\end{comment}