\begin{abstractind}
Pada Kota Banda Aceh, pemahaman tentang sistem hidroponik pada Kota ini masih tergolong baru. Proses bisnis yang dilakukan oleh petani hidroponik Banda Aceh untuk melakukan penjualan tanaman mereka masih menggunakan media sosial seperti Instagram, dan WhatsApp. Dengan menggunakan media sosial Instagram atau WhatsApp tersebut, pendapatan atau penjualan yang mereka dapatkan sebagian besar hanya dari komunitas yang mereka ikuti. Dari hasil wawancara dengan beberapa petani hidroponik di Kota Banda Aceh diketahui bahwa mereka membutuhkan solusi untuk menjual dan memperluas pemasaran tanaman hidroponik mereka. Penelitian ini akan merancang dan membangun aplikasi marketplace dengan menggunakan metode Scrum. Tujuan dari penyusunan penelitian ini adalah menghasilkan aplikasi marketplace untuk petani hidroponik berbasis web di Kota Banda Aceh. Aplikasi yang telah dibangun dapat menjadi sebuah media atau perantara proses bisnis antara penjual dan petani hidroponik.


\bigskip
\noindent
\textbf{Kata kunci :} marketplace hidroponik, aplikasi berbasis web, laravel, php
\end{abstractind}