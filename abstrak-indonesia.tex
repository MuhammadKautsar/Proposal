\begin{abstractind}

Pelaku UMKM hidroponik di Banda Aceh telah memanfaatkan aplikasi media sosial untuk memasarkan produknya. Namun sayangnya cara ini masih memiliki kekurangan seperti akses yang terbatas, calon pembeli tidak mengetahui akun media sosial penjual dan juga ketersediaan sayuran hidroponik yang susah terdeteksi. Oleh karena itu, penelitian ini menawarkan solusi atas permasalahan tersebut berupa aplikasi berbasis web yang dapat membantu pelaku UMKM hidroponik untuk memasarkan produknya dengan lebih mudah dan efisien. Pada aplikasi ini terdapat 3 tipe \textit{user} yaitu superadmin, admin dan penjual. \textit{User} superadmin dan admin dapat menggunakan aplikasi untuk mengelola data, akan tetapi terdapat perbedaan di mana superadmin bertugas mendaftarkan akun admin sedangkan admin bertugas mendaftarkan akun penjual, lalu penjual dapat menggunakan aplikasi ini untuk menjual produknya dan menerima pesanan dari pembeli. Proses rancang bangun aplikasi dimulai dari tahap analisis kebutuhan, perancangan sistem, kemudian dilanjutkan dengan pembuatan aplikasi dan pengujian aplikasi. Pembuatan aplikasi penjualan berbasis web ini dibangun menggunakan \textit{framework} Laravel dan MySQL sebagai tempat menyimpan \textit{database}. Pada aplikasi ini juga dibuatkan REST API untuk digunakan sebagai \textit{backend} pada aplikasi berbasis Android. Setelah aplikasi selesai dibuat, dilakukan pengujian fungsionalitas menggunakan metode \textit{Black Box} dan pengujian \textit{usability} menggunakan \textit{Usability Metric for User Experience} (UMUX). Hasil dari pengujian fungsionalitas diperoleh bahwa aplikasi berhasil menjalankan fungsi yang diuji. Sedangkan hasil dari pengujian \textit{usability} menghasilkan nilai 82,12 dari kepuasan pengguna.

% Pengujian fungsionalitas menghasilkan nilai yang ’sesuai’ untuk seluruh fungsi yang diuji. Sedangkan hasil dari pengujian \textit{usability} menghasilkan nilai 82,12 dari kepuasan pengguna.

% Namun, perbedaan antara hasil UMUX dan dua skala lainnya cukup besar untuk mengarahkan praktisi membuat keputusan yang berbeda, di mana besarnya rata-rata UMUX cukup dekat dengan rata-rata SUS yang sesuai, dalam penelitian ini nilai rata-rata UMUX secara signifikan lebih tinggi dibandingkan dengan rata-rata SUS dan UMUX-LITE. Pengujian fungsionalitas menghasilkan nilai yang 'sesuai' dan begitu pula hasil dari pengujian \textit{usability} menghasilkan nilai 82.12\% dari kepuasan pengguna.

% Namun, karena hasil UMUX dinilai terlalu positif dan belum memiliki \textit{grading scale} sendiri, maka skor UMUX yang diperoleh dikonversi ke UMUX-lite yang lebih mendekati skor SUS agar dapat dilakukan penilaian terhadap hasil \textit{usability testing} yang dilakukan. Pengujian fungsionalitas menghasilkan nilai yang sesuai dengan yang diharapkan dan hasil dari pengujian \textit{usability} menghasilkan nilai 82.12\% yang berarti aplikasi dapat diterima dan layak digunakan.

\bigskip
\noindent
\textbf{Kata kunci :} UMKM, Hidroponik, Laravel, MySQL, \textit{Black Box, Usability Metric for User Experience.}
\end{abstractind}