\begin{abstractind}

Perkembangan teknologi informasi yang pesat telah memberikan banyak manfaat dalam berbagai aspek kehidupan, tak terkecuali dibidang UMKM. Dimana para pelaku UMKM saat ini telah memanfaatkan teknologi informasi dalam memasarkan produknya secara digital, diantaranya menggunakan aplikasi media sosial. Hal inilah yang dilakukan oleh para kelompok UMKM tanaman hidroponik yang ada di kota Banda Aceh, hanya saja cara ini dinilai masih memiliki beberapa kekurangan yaitu diantaranya akses yang terbatas antara calon pembeli dengan penjual dikarenakan tidak semua calon pembeli mengetahui akun media sosial penjual dan juga pembeli tidak dapat mengetahui dengan pasti ketersediaan produk yang ditawarkan oleh penjual. Oleh karena itu, pada penelitian ini menawarkan solusi atas permasalahan tersebut berupa aplikasi berbasis web yang dapat membantu UMKM hidroponik untuk memasarkan produknya dengan lebih mudah dan efesien. Pada aplikasi ini terdapat 2 tipe \textit{user} yaitu admin dan penjual. Dimana admin dapat menggunakan aplikasi untutk mengelola data dan mendaftarkan akun penjual, sedangkan penjual dapat menggunakan aplikasi ini untuk menjual produknya dan menerima pesanan dari pembeli. Proses rancang bangun aplikasi dimulai dari tahap analisis kebutuhan, perancangan sistem, kemudian dilanjutkan dengan pembuatan aplikasi dan pengujian aplikasi. Pembuatan aplikasi penjualan berbasis web ini dibangun menggunakan \textit{framework} Laravel dan MySQL sebagai \textit{database}. Pada aplikasi ini juga dibuatkan REST API untuk digunakan sebagai \textit{backend} pada aplikasi berbasis Android. Setelah aplikasi selesai dibuat, dilakukan pengujian fungsionalitas menggunakan metode \textit{Black Box} dan pengujian \textit{usability} menggunakan \textit{Usability Metric for User Experience} (UMUX). Pengujian fungsionalitas menghasilkan nilai yang 'sesuai' dan begitu pula hasil dari pengujian UMUX menghasilkan nilai 86.66\% dari kepuasan pengguna.


\bigskip
\noindent
\textbf{Kata kunci :} UMKM, Hidroponik, web, Laravel, \textit{framework}, MySQL, \textit{Black Box, Usability Metric for User Experience.}
\end{abstractind}