%-------------------------------------------------------------------------------
%                            BAB V
%               		KESIMPULAN DAN SARAN
%-------------------------------------------------------------------------------
\fancyhf{} 
\fancyfoot[C]{\thepage}
\chapter{KESIMPULAN DAN SARAN}

\section{\uppercase{KESIMPULAN}}
Berdasarkan penelitian yang telah dilakukan, dapat diambil kesimpulan bahwa:
\begin{enumerate}
    \item Hasil perancangan menghasilkan beberapa diagram seperti \textit{use case} diagram, \textit{sequence} diagram, \textit{activity} diagram, dan \textit{entity relationship} diagram. Hasil rancangan telah diimplementasikan dengan menghasilkan aplikasi berbasis web yang dapat digunakan oleh admin dan penjual tanaman hidroponik di kota Banda Aceh.
    \item Laravel sebagai \textit{framework} pembuatan aplikasi web telah berhasil diimplementasikan pada penelitian ini untuk membangun aplikasi penjualan tanaman hidroponik berbasis web. Aplikasi yang dibangun ini diberi nama Agrihub.
    \item Penelitian ini telah berhasil membangun REST API dari aplikasi berbasis web untuk digunakan sebagai \textit{backend} pada aplikasi berbasis Android. REST API ini digunakan sebagai penghubung antara kedua aplikasi tersebut.
    % \item Aplikasi berbasis web menyediakan REST API untuk digunakan pada aplikasi berbasis android sebagai penghubung antara keduanya.
    % \item Aplikasi berbasis web ini hanya ditujukan untuk admin dan penjual. Adapun admin bisa melakukan aktivitas daftar akun penjual di dalam aplikasi berbasis web. Penjual dapat melakukan masuk akun melalui aplikasi web dan android.
    % \item Aplikasi yang dibangun saat ini hanya berfokus pada wilayah kota Banda Aceh.
    \item Berdasarkan hasil pengujian fungsionalitas menggunakan metode \textit{Black Box}, aplikasi penjualan tanaman hidroponik ini telah berjalan sesuai dengan yang diharapkan.
    \item Berdasarkan hasil pengujian \textit{usability} menggunakan metode UMUX, aplikasi penjualan tanaman hidroponik ini memiliki rata-rata skor sebesar 86,66 dan skor UMUX-lite sebesar 82,12 yang berarti aplikasi dapat diterima dan layak digunakan.
\end{enumerate}

\section{\uppercase{SARAN}}
Adapun beberapa saran yang dapat digunakan untuk peningkatan aplikasi, diantaranya sebagai berikut:
\begin{enumerate}
    \item Tampilan antarmuka aplikasi dibuat lebih bagus dan mudah digunakan seperti aplikasi Tokopedia.
    \item Memanfaatkan fitur jarak sehingga penjual tidak perlu mengatur harga ongkos kirim secara manual.
    \item Bekerja sama dengan penyedia jasa kurir agar status pesanannya bisa \textit{terupdate} secara otomatis di sistem.
    % \item Bekerja sama dengan penyedia jasa kurir agar status proses pengiriman pesanan bisa berubah otomatis disistem.
    % \item Bekerja sama dengan penyedia jasa kurir agar pesanan tidak perlu dikirim oleh penjual sendiri dan bisa otomatis disistem.
    % \item Dapat diperluas lagi penggunaan aplikasi tidak hanya di kota Banda Aceh.
\end{enumerate}

%-----------------------------------------------------------------------------%

% Baris ini digunakan untuk membantu dalam melakukan sitasi
% Karena diapit dengan comment, maka baris ini akan diabaikan
% oleh compiler LaTeX.
\begin{comment}
\bibliography{daftar-pustaka}
\end{comment}