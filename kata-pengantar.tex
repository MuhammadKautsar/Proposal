\preface % Note: \preface JANGAN DIHAPUS!


Segala puji dan syukur atas kehadirat Allah SWT yang telah melimpahkan rahmat dan karunia-Nya kepada kita semua, sehingga penulis dapat menyelesaikan penulisan tugas akhir ini yang berjudul \textbf{“Rancang Bangun Apikasi Penjualan Tanaman Hidroponik Berbasis Web (Studi Kasus : Kelompok Penjual Hidroponik Banda Aceh)”}. Penulis menyadari penulisan tugas akhir ini tidak terlepas dari dukungan, pengarahan, bimbingan, dan bantuan dari berbagai pihak. Oleh karena itu, melalui tulisan ini penulis mengucapkan rasa terima kasih kepada:

\begin{enumerate}
	\item{Ayah dan Ibu sebagai kedua orang tua penulis yang senantiasa selalu mendukung aktivitas dan kegiatan yang penulis lakukan baik secara moral maupun material serta menjadi motivasi terbesar bagi penulis untuk menyelesaikan Proposal ini.}
	\item{Bapak Kurnia Saputra, M.Sc., selaku Dosen Pembimbing I dan Ibu Viska Mutiawani, B.IT, M.IT., selaku Dosen Pembimbing II yang telah banyak memberikan bimbingan dan arahan kepada penulis, sehingga penulis dapat menyelesaikan Tugas akhir ini.}
	\item {Bapak Dr. Muhammad Subianto, M.Si., selaku Ketua Jurusan Informatika.}
	\item{Bapak Zahnur S.Si, M.Info Tech., selaku Dosen Wali.}
	\item{Seluruh Dosen di Jurusan Informatika Fakultas MIPA atas ilmu dan didikannya selama perkuliahan.}
	\item{Sahabat dan teman-teman seperjuangan Jurusan Informatika Unsyiah 2016 lainnya.}
\end{enumerate}

% \vspace{6cm}

Penulis juga menyadari segala ketidaksempurnaan yang terdapat didalamnya baik dari segi materi, cara, ataupun bahasa yang disajikan. Seiring dengan ini penulis mengharapkan kritik dan saran dari pembaca yang sifatnya dapat berguna untuk kesempurnaan Proposal ini. Harapan penulis semoga tulisan ini dapat bermanfaat bagi banyak pihak dan untuk perkembangan ilmu pengetahuan.

\vspace{1cm}


\begin{tabular}{p{7.5cm}c}
	&Banda Aceh, September 2020\\
	&\\
	&\\
	&\\
	&\textbf{Penulis}
\end{tabular}