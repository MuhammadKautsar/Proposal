\begin{abstracteng}
\textit{The rapid development of information technology has provided many benefits in various aspects of life, not least in the field of UMKMs. Where SMEs are currently using information technology in marketing their products digitally, including using social media applications. This is what hydroponic plant UMKM groups are doing which is in the city of Banda Aceh, it's just that this method is considered to still have several shortcomings, including limited access between prospective buyers and sellers because not all prospective buyers know the seller's social media account and also buyers cannot know for sure the availability of products offered by the seller. Therefore, this study offers a solution to these problems in the form of a web-based application that can help hydroponic SMEs to market their products more easily and efficiently. In this application there are 2 types of users, namely admin and seller. Where admins can use the application to manage data and register seller accounts, while sellers can use this application to sell their products and receive orders from buyers. The application design process starts from the requirements analysis stage, system design, then continues with application creation and application testing. Making a web-based sales application is built using the Laravel framework and MySQL as a database. In this application, a REST API will also be created to be used as a backend for Android-based applications. After the application is completed, functional testing is carried out using the Black Box method and usability testing using the Usability Metric for User Experience (UMUX). The functionality test yielded an 'appropriate' value and similarly the results from the UMUX test yielded a value of 86.66\% of user satisfaction.}

\bigskip
\noindent
\textbf{\emph{Keywords :}} \textit{UMKM, Hydroponic, web, Laravel, framework, MySQL, Black Box, Usability Metric for User Experience.}
\end{abstracteng}