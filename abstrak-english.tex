\begin{abstracteng}
\textit{Hydroponic MSME actors in Banda Aceh have used social media applications to market their products. But unfortunately this method still has drawbacks such as limited access, potential buyers not knowing the seller's social media accounts and also the availability of hydroponic vegetables which are difficult to detect. Therefore, this study offers a solution to these problems in the form of a web-based application that can help hydroponic SMEs to market their products more easily and efficiently. In this application there are 3 types of users, namely superadmin, admin and seller. Superadmin and admin users can use the application to manage data, but there is a difference where superadmin is in charge of registering an admin account while the admin is in charge of registering a seller's account, then sellers can use this application to sell their products and receive orders from buyers. The application design process starts from the requirements analysis stage, system design, then continues with application creation and application testing. Making a web-based sales application is built using the Laravel framework and MySQL as a place to store the database. In this application, a REST API is also made to be used as a backend for Android-based applications. After the application is completed, functional testing is carried out using the Black Box method and usability testing using the Usability Metric for User Experience (UMUX). Functionality testing returns 'fit' values for all tested functions. While the results of usability testing produce a value of 82.12 from user satisfaction.}

\bigskip
\noindent
\textbf{\emph{Keywords :}} \textit{MSME, Hydroponic, Laravel, MySQL, Black Box, Usability Metric for User Experience.}
\end{abstracteng}